\documentclass[10pt, compress]{beamer}

\usetheme{m}

\usepackage{booktabs}
\usepackage[scale=2]{ccicons}
\usepackage{minted}

\usepgfplotslibrary{dateplot}

\usemintedstyle{trac}

\title{Environnements de MOOC avec Docker}
\subtitle{Projet de fin d'études - VAP ASR}
\date{28 Janvier 2015}
\author{François Monniot \& Alexis Mousset}
\institute{Télécom SudParis}

\begin{document}

\maketitle

\section{Introduction}

\begin{frame}[fragile]
  \frametitle{Notre projet}

  Exemple de \emph{code} (du \alert{beau}) :

  \begin{minted}[fontsize=\footnotesize]{javascript}
// Instanciation du tracker
var tracker = new Tracker({distant: 'url'});

// On ecoute les evenements "onclick" sur les elements de classe "my-element"
tracker.on('.my-element').track('click');
  \end{minted}
  \begin{description}
    \item[PowerPoint] Meeh.
    \item[Beamer] Yeeeha.
  \end{description}
\end{frame}

\section{Présentation}

\section{Cas d'utilisation}

\section{Démonstration}

\plain{Questions ?}

\end{document}
